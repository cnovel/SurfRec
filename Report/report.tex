\documentclass[a4paper]{article}

\usepackage[english]{babel}
\usepackage[utf8]{inputenc}
\usepackage{amsmath}
\usepackage{graphicx}

\title{Surface reconstruction\\
\large{CO512 ISO Report}}

\author{Cyril NOVEL}

\date{\today}

\begin{document}
\maketitle
\newpage

\tableofcontents

\newpage

\section*{Introduction}
\addcontentsline{toc}{section}{Introduction}
In the last decades, numerous devices have been developped in order to capture and digitize our world. Among them, 3D scanners aim at capturing the geometry of the objects surounding us. Statues, monuments but also industrial parts, landscapes : all these can be digitized into a 3D point cloud. But a 3D point cloud isn't enough for a correct visualization by human beings. Thus the reconstruction of the surface is needed to obtain a more comprehensible object.

Because of the numerous fields it covers -- cultural, industrial, medical -- Surface reconstruction is a widely studied subject in Computer Science. In this report, we will present the three main categories of algorithms used to perform surface reconstruction. For each category we will describe two or more precise algorithms.

\newpage

\section{Delaunay-based algorithms}
Delaunay-based algorithms are based on the Delaunay triangulation of a point cloud and the Voronoi diagram of a point cloud. In 2D, a Delaunay triangulation of a set of points $P$ is a triangulation such that no point in $P$ is inside the circumcircle of any triangle of the triangulation. We can generalize Delaunay triangulation in a $n$-dimension space, using simplex. For surface reconstruction, $n = 3$ so that the property becomes : a Delaunay triangulation of a set of points $P$ such that no point in $P$ is inside the circumspeher of any tetrahedron of the triangulation.

A Voronoi diagram is a way to divide the space. Given a set of points $\{p_i\}$,  for each point there will be a corresponding region $R_i = \{x | dist(p_i,x) < dist(p_j,x) ; i \ne j\}$. The regions are called Voronoi cells.

The Voronoi diagram is the dual of the Delaunay triangulation.

\subsection{General idea}
The Delaunay-based algorithm can be divided into two main steps. First, it computes a geometric triangulation of the finite set of points via the Delaunay triangulation or the Voronoi diagram. Then, it extracts a collection of facets chosen so that they are close to the actual surface.

The advantage of Delaunay-based algorithms is that the triangulation created is more robust than other approaches. Delaunay triangulation comes with a variety of theoretical guarantees. However, the Delaunay triangulation is very expensive to compute, thus these algorithms are very expensive for large point clouds.

We describe two different Delaunay based algorithms : the Power Crust algorithm and the Cocone algorithm.
\subsection{Power crust algorithm}
The Power crust construct
The power cr, st is a construction which takes a sample of points
fiom the surface of a three-dimensional object and produces a sur-
lace mesh and an approximate medial axis. The approach is to first
approximate the medial axis transform (MAT) o f the object. We
then use an inverse transform to produce the surface representation
fi'om the MAT.
\subsection{Cocone algorithm}
\newpage

\section{Region-growing algorithms}
\subsection{General idea}
A classic region-growing algorithm begins by initiating a triangle as an initial region and then iterates to link new triangles on the region's boundaries. This type of algorithm is very fast. Most of the time, each point of the point cloud is considered only once for the triangulation. The disadvantage is that the reconstruction relies heavily on parameters chosen by the user and on the sampling of the point cloud. Moreover this method can create small holes in the surface when poor data  exists -- due to noise for example. 

We describe two different region-growing algorithms : the Ball Point algorithm and the Fast Reconstruction algorithm of the Point Cloud Library.

\subsection{Ball Point algorithm}
The idea of the BPA is very simple : three points form a triangle if a ball of radius \rho touches them without containing any other points. Starting with a seed triangle, the sphere pivots around an edge of the triangle until it touches another point. It then formed a new triangle. We loop this process until all reachable edges have been considered.


\subsection{Fast Reconstruction algorithm (PCL)}
\newpage

\section{Implicit surface algorithms}
\subsection{General idea}
In an implicit surface algorithm, a signed distance function is defined from the sample points and computed. Then given the zero-set of the signed distance function, an approximate triangulated surface is constructed.

There is a major difference between Implicit surface algorithms and Region-growing or Delaunay-based algorithms. Region-growing and Delaunay based methods interpolate sample points, mean that the reconstructed surface lies on the sample points. The implicit surface approach approximates sample points and can lack accuracy. But one advantage is that implicit surface methods are ideal in the case of highly noised data.

We describe four different implicit surface algorithms : the Poisson algorithm, the Radial Basis Function algorithm, the Level Set algorithm and the Marching Cubes algorithm.


\subsection{Poisson algorithm}

\subsection{Radial Basis Function algorithm}

\subsection{Level Set algorithm}

\subsection{Marching cubes algorithm}


\section*{Conclusion}
\addcontentsline{toc}{section}{Conclusion}
\end{document}
