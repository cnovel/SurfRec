\documentclass[a4paper]{article}

\usepackage[english]{babel}
\usepackage[utf8]{inputenc}
\usepackage{amsmath}
\usepackage{graphicx}

\title{Surface reconstruction\\
\large{CO512 ISO Report}}

\author{Cyril NOVEL}

\date{\today}

\begin{document}
\maketitle
\newpage

\tableofcontents

\newpage

\section*{Introduction}
\addcontentsline{toc}{section}{Introduction}
In the last decades, numerous devices have been developped in order to capture and digitize our world. Among them, 3D scanners aim at capturing the geometry of the objects surounding us. Statues, monuments but also industrial parts, landscapes : all these can be digitized into a 3D point cloud. But a 3D point cloud isn't enough for a correct visualization by human beings. Thus the reconstruction of the surface is needed to obtain a more comprehensible object.

Because of the numerous fields it covers -- cultural, industrial, medical -- Surface reconstruction is a widely studied subject in Computer Science. In this report, we will present the three main categories of algorithms used to perform surface reconstruction. For each category we will describe two or more precise algorithms.

\newpage

\section{Delaunay-based algorithms}
Delaunay-based algorithms are based on the Delaunay triangulation of a point cloud and the Voronoi diagram of a point cloud. In 2D, a Delaunay triangulation of a set of points $P$ is a triangulation such that no point in $P$ is inside the circumcircle of any triangle of the triangulation. We can generalize Delaunay triangulation in a $n$-dimension space, using simplex. For surface reconstruction, $n = 3$ so that the property becomes : a Delaunay triangulation of a set of points $P$ such that no point in $P$ is inside the circumspeher of any tetrahedron of the triangulation.

A Voronoi diagram is a way to divide the space. Given a set of points $\{p_i\}$,  for each point there will be a corresponding region $R_i = \{x | dist(p_i,x) < dist(p_j,x) ; i \ne j\}$. The regions are called Voronoi cells.

The Voronoi diagram is the dual of the Delaunay triangulation.

\subsection{General idea}
The Delaunay-based algorithm can be divided into two main steps. First, it computes a geometric triangulation of the finite set of points via the Delaunay triangulation or the Voronoi diagram. Then, it extracts a collection of facets chosen so that they are close to the actual surface.

The advantage of Delaunay-based algorithms is that the triangulation created is more robust than other approaches. Delaunay triangulation comes with a variety of theoretical guarantees. However, the Delaunay triangulation is very expensive to compute, thus these algorithms are very expensive for large point clouds.

We describe two different Delaunay based algorithms : the Power Crust algorithm and the Cocone algorithm.

\subsection{Power crust algorithm}
The Powercrust algorithm compute an approximate medial axis of the 3D point cloud $S$ using the Voronoi diagram. We then take a subset $V$ of the Voronoi vertices, called the poles. The poles lie near the medial axis. Each pole has an associated ball, called polar ball. The polar balls approximate maximal balls contained in the interior or exterior of the 3D shape. The radii of the polar balls are the weights of the poles. The inverse transform is then approximate using the power diagram using the set of weighted poles.

The Power Crust algorithm can be summarize in 6 steps :
\begin{enumerate}
\item Compute the Voronoi diagram of the sample points S.
\item For each sample point, compute its poles.
\item Compute the power diagram of the poles.
\item Label each pole either inside or outside.
\item Output the power diagram faces separating the cells of inside and outside poles as the power crust.
\item Output the regular triangulation faces connecting inside poles as the power shape.
\end{enumerate}

The Power crust algorithm uses the notion of the \textit{Medial Axis Transform}. Let $F$ be the boundary of the 3D object. To avoid infinity edges and/or points, we assume that $F$ is contained in a bounded open region $Q$ -- for example a 3D box containing $F$. $F$ divides $Q$ into interior and exterior solids. Let $B_{c,r}$ be a ball of center $c$ and radius $r$. $B$ is empty if the interior of $B$ contains no point of $F$. A medial ball is a maximal empty ball, meaning that no other empty ball can completely contain it. The center of this medial ball is either a center of curvature of $F$ or a point with more than one closest point on $F$.

The \textit{Medial Axis Transform} of $F$ is the set of medial balls of F. The centers of the medial balls form the medial axis of $F$. The medial axis includes a part inside of $F$ and a part outside of $F$. The medial axis of a three-dimensional object is generally a two-dimensional surface.

Assuming a dense sampling, the Voronoi cell of every point of the data set is long, skinny and \textit{perpendicular} to the surface. It happens because in directions tangent to the surface the Voronoi cell is limited by the very close neighbors. So the Voronoi cell extends perpendicularly away from the surface and cannot extend farther than the medial axis. There it is not the closest point anymore -- due to the nature of the medial axis. Thus, the Voronoi vertices at the two ends of the long, skinny Voronoi cell should lie near the median axis. This motivates the selection of poles as an approximation of the median axis.

The inverse transform uses a \textit{power diagram}. A finite set of balls can be related to power diagrams, which are a kind of Voronoi diagrams, using the distance function :
$$d_{pow}(x, B_{c,r}) = d^2(c,x) - r^2$$
with $d$ the usual distance function in an euclidean space  $B_{c,r}$ the ball of center $c$ and radius $r$. Note that if $x$ is inside $B_{c,r}$, $d_{pow}$ is negative. We use $d_{pow}$ to define a Voronoi diagram, called the \textit{power diagram}.

Considering the power diagram of the polar balls, the \textit{power crust} is the boundary between the power diagrams cells belonging to inner poles and power diagramms cells belonging to outer poles. A facet of the power crust separates cells corresponding to an inner and an outer pole. With a dense sampling, the two polar balls should intersect only slightly, since the inner ball is mostly inside the object and the outer ball outside. The power crust actually interpolates the points of the original point cloud.



Since most points of the interior solid bounded by F are inside the
union of the inner polar balls, and outside of the union of outer po-
lar balls, they belong to cells o f the power diagram corresponding
to inner poles. Similarly most points in the exterior solid belong to
cells corresponding to outer poles.

3.4    Power shape
The definition of the power crust implies a way to connect the poles
to form a topologically correct [3] approximation o f the medial axis
as a simplicial complex M, which we call the power shape. The ver-
tices of M are the poles themselves. Inner poles whose cells are ad-
jacent in the power diagram are connected by simplices in M, as are
adjacent outer poles. The power shape is a subset of the weighted
Delaunay triangulation ( also known as the regular triangulation)
dual to the power diagram, just as the Delaunay triangulation is dual
to the usual unweighted Voronoi diagram. While the medial axis of
F is a two-dimensional surface, the power shape generally contains
 some very fiat, but solid, tetrahedra.

Steps 1 and 3 of the Power crust algorithm are dependent on Voronoi diagram algorithms and won't be detailed here.


\subsection{Cocone algorithm}

\newpage

\section{Region-growing algorithms}
\subsection{General idea}
A classic region-growing algorithm begins by initiating a triangle as an initial region and then iterates to link new triangles on the region's boundaries. This type of algorithm is very fast. Most of the time, each point of the point cloud is considered only once for the triangulation. The disadvantage is that the reconstruction relies heavily on parameters chosen by the user and on the sampling of the point cloud. Moreover this method can create small holes in the surface when poor data  exists -- due to noise for example. 

We describe two different region-growing algorithms : the Ball Point algorithm and the Fast Reconstruction algorithm of the Point Cloud Library.

\subsection{Ball Point algorithm}
The idea of the BPA is very simple : three points form a triangle if a ball of radius $\rho$ touches them without containing any other points. Starting with a seed triangle, the sphere pivots around an edge of the triangle until it touches another point. It then formed a new triangle. We loop this process until all reachable edges have been considered.


\subsection{Fast Reconstruction algorithm (PCL)}
\newpage

\section{Implicit surface algorithms}
\subsection{General idea}
In an implicit surface algorithm, a signed distance function is defined from the sample points and computed. Then given the zero-set of the signed distance function, an approximate triangulated surface is constructed.

There is a major difference between Implicit surface algorithms and Region-growing or Delaunay-based algorithms. Region-growing and Delaunay based methods interpolate sample points, mean that the reconstructed surface lies on the sample points. The implicit surface approach approximates sample points and can lack accuracy. But one advantage is that implicit surface methods are ideal in the case of highly noised data.

We describe four different implicit surface algorithms : the Poisson algorithm, the Radial Basis Function algorithm, the Level Set algorithm and the Marching Cubes algorithm.


\subsection{Poisson algorithm}

\subsection{Radial Basis Function algorithm}

\subsection{Level Set algorithm}

\subsection{Marching cubes algorithm}


\section*{Conclusion}
\addcontentsline{toc}{section}{Conclusion}
\end{document}
