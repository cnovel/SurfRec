\documentclass[a4paper]{article}

\usepackage[english]{babel}
\usepackage[utf8]{inputenc}
\usepackage{amsmath}
\usepackage{graphicx}

\title{Surface reconstruction\\
\large{CO512 ISO Report}}

\author{Cyril NOVEL}

\date{\today}

\begin{document}
\maketitle
\newpage

\section{Introduction}
In the last decades, numerous devices have been developped in order to capture and digitize our world. Among them, 3D scanners aim at capturing the geometry of the objects surounding us. Statues, monuments but also industrial parts, landscapes : all these can be digitized into a 3D point cloud. But a 3D point cloud isn't enough for a correct visualization by human beings. Thus the reconstruction of the surface is needed to obtain a more comprehensible object.
Because of the numerous fields it covers -- cultural, industrial, medical -- Surface reconstruction is a widely studied subject in Computer Science. In this report, we will present the three main categories of algorithms used to perform surface reconstruction. For each category we will describe two or more precise algorithms.

\section{Delaunay-based algorithms}
% TO DO
%  - What is a Delaunay triangulation / Voronoi diagram
%  - Describe the idea
%  - Describe Power crust
%  - Describe Cocone

\section{Region-growing algorithms}
% TO DO
%  - What is a Region growing algo
%  - Describe Ball Point Algorithm
%  - Describe Fast Algo from PCL

\section{Implicit surface algorithms}
% TO DO
%  - What is implicit surface algo
%  - Describe Poisson
%  - Describe Radial Basis Function
%  - Describe Level Set
%  - Describe Marching Cubes


\section{Conclusion}
\end{document}
